\documentclass[a4paper,11pt,twoside]{scrartcl}
\usepackage{ngerman, eucal, mathrsfs, amsfonts, bbm, amsmath, amssymb, stmaryrd,graphicx, array, xcolor, ulem, epstopdf}
\usepackage[T1]{fontenc}
\usepackage[utf8]{inputenc}
\usepackage{geometry}
\geometry{left=25mm, right=15mm, bottom=25mm}
\setlength{\parindent}{0em} 
\setlength{\headheight}{0em} 
\newcommand{\korr}[2]{\sout{#1} \textcolor{red}{\underline{#2}}}
\usepackage{xcolor}
\usepackage{listings}

\definecolor{comment_green}{rgb}{0, 0.5, 0}
\newcommand{\green}[1]{\textcolor{comment_green}{#1}}
\newcommand{\red}[1]{\textcolor{red}{#1}}

\lstset{%
	mathescape = true,%
	literate=%
	{Ö}{{\"O}}1
	{Ä}{{\"A}}1
	{Ü}{{\"U}}1
	{ß}{{\ss}}2
	{ü}{{\"u}}1
	{ä}{{\"a}}1
	{ö}{{\"o}}1
	{pi}{{$\Pi$}}1
}

\title{Theoretische Grundlagen der Informatik II\\ Blatt 10}
\author{Markus Vieth \and Marvin Becker}
\date{\today}

\newcounter{beweis}

\begin{document}

\maketitle
\cleardoublepage
\pagestyle{myheadings}
\markboth{Markus Vieth, Marvin Becker}{Markus Vieth, Marvin Becker}

\section*{Aufgabe 1}
\paragraph{Pseudocode}
Sei \texttt{E} das Ergebnis des Monte Carlo Algorithmus, \texttt{I} die Eingabe und \texttt{boolean test(Ergebnis E)} der Algorithmus, welcher das Ergebnis überprüft.
\begin{lstlisting}
Ergebnis E;
do {
	E = MonteCarloAlgorithmus(I); 
}while(!test(E));
return E;
\end{lstlisting}
\paragraph{Laufzeit}
Sei $q(n)=1-p(n)$ und $P(X = i) := q(n)^{i-1}\cdot p(n)$ die Wahrscheinlichkeit, dass der Monte Carlo Algorithmus nach $i$ Durchläufen das richtige Ergebnis liefert (also zuerst $i-1$ falsche und dann ein richtiges). Dann beträgt die zu erwartende Anzahl an Durchläufen:
\[ E(X)=\sum_{i=1}^{\infty} i\cdot q(n)^{i-1}\cdot p(n)=\footnote{$\sum_{i=1}^{\infty} i\cdot q(n)^{i-1}\cdot p(n)$ ist absolut konvergent, wenn $0<p(n)<1$, kann per Quotientenkriterium gezeigt werden}p(n)\cdot\sum_{i=1}^{\infty} i\cdot q(n)^{i-1}=\footnote{Ableitung der geometrischen Reihe} p(n)\cdot\frac{1}{(1-q(n))^2}=\frac{p(n)}{(1-1+p(n))^2}=\frac{1}{p(n)}  \]
Sei $T_{LV}(n)$ die Laufzeit des oben beschriebenen Algorithmus:
\[ \Rightarrow T_{LV}(n)=E(X)\cdot (T(n)+t(n))=\frac{T(n)+t(n)}{p(n)}  \]
\begin{flushright}
	q.e.d.
\end{flushright}
\korr{}{Sei $X$ die Anzahl der Durchläufe}
\korr{}{$ X \~ \text{geom}_p \Rightarrow E[X] = \frac{1}{p(n)} $} 
\korr{}{\[ P(X=k)=(1-p)^{k-1}\cdot p \]
	\[ E[X] = \sum_{k = 1}^{\infty} = \sum_{k = 1}^{\infty}(1-p)^{k-1}pk = p\sum_{k = 1}^{\infty}(1-p)^{k-1}k = p \sum_{n=0}^{\infty}\frac{-d}{dp}(1-p)^k \]}
\section*{Aufgabe 2}
\subsection*{a)}
\paragraph{Behauptung}
Wenn $f$ und $g$ vernachlässigbar sind, dann ist $f + g$ vernachlässigbar.
\paragraph{Beweis}
Sei $\mathcal{P}$ die Menge aller Polynome,
\[ f, g \text{ vernachlässigbar } \Rightarrow\forall p(x)\in\mathcal{P} \exists N\in\mathbb{N} | f(n)<\frac{1}{2p(n)} \land g(n) < \frac{1}{2p(n)} \forall n>N\]
\[ \Rightarrow f(n)+g(n) < \korr{\frac{1}{2p}+\frac{1}{2p}=\frac{2}{2p}=\frac{1}{p}}{\frac{1}{2p(n)}+\frac{1}{2p(n)}=\frac{2}{2p(n)}=\frac{1}{p(n)}~\forall n>N} \]
\begin{flushright}
	q.e.d.
\end{flushright}
\subsection*{b)}
\paragraph{Behauptung}
Wenn $f$ und $g$ vernachlässigbar sind, dann ist $f g$ vernachlässigbar.
\paragraph{Beweis}
\[ f,g \text{ vernachlässigbar } \Rightarrow\footnote{da 1 ein Polynom vom Grad 0 ist} \korr{f<1}{f(n)<1} \land \korr{g<1}{g(n)<1} \Rightarrow \korr{fg<g<p(n)}{f(n)g(n)<g(n)<\frac{1}{p(n)}}\Rightarrow \korr{fg}{f(n)g(n)} \text{ ist vernachlässigbar} \]
\begin{flushright}
	q.e.d.
\end{flushright}
\subsection*{c)}
\paragraph{Behauptung}
Wenn $f$ vernachlässigbar und $g$ eine beliebige positive Funktion ist, dann ist $f + g$
vernachlässigbar.
\paragraph{Gegenbeispiel}
\subparagraph{Anmerkung:} Im folgenden wird davon ausgegangen, dass vernachlässigbare Funktionen stets größer oder gleich 0 sind.\\
Sei $f$ eine beliebige vernachlässigbare Funktion und $g=1$
\[ \not\exists N\in\mathbb{N} ~|~ f(n)+1 < 1~\forall~n>N \]
\begin{flushright}
	q.e.d.
\end{flushright}
\subsection*{d)}
\paragraph{Behauptung}
Wenn $f$ vernachlässigbar und $c$ eine positive Konstante ist, dann ist $cf$ vernachlässigbar.
\paragraph{Beweis}
\[ f \text{ vernachlässigbar }\Rightarrow \forall p(x)\in\mathcal{P} \exists N\in\mathbb{N} | f(n)<\frac{1}{c\cdot p(n)} \forall n>N\Rightarrow c\cdot f(n) < \frac{c}{c\cdot p(n)} = \frac{1}{p(n)}\]
\[ \Rightarrow cf \text{ ist vernachlässigbar.} \]
\begin{flushright}
	q.e.d.
\end{flushright}
\subsection*{e)}
\paragraph{Behauptung}
$f(n) = \frac{1}{n^{10}}$. $f$ ist vernachlässigbar.
\paragraph{Gegenbeispiel}
\[ \not\exists N\in\mathbb{N} | \frac{1}{n^{10}} < \frac{1}{n^{42}} \Rightarrow f\text{ ist nicht vernachlässigbar.} \]
\begin{flushright}
	q.e.d.
\end{flushright}
\section*{Aufgabe 3}
\subsection*{a)}

\paragraph{Behauptung}  Sei $H_1(s) = G_2(\overline{s})$.  Zeige, dass $H_1$ ein PZG ist.
\paragraph{Beweis}
\paragraph{Zu zeigen:$\ell_{H_1}(n) > n$}
\[ H_1(s) = G_2(\overline{s}) \land \ell_{G_2}(n) = 2n \Rightarrow \ell_{H_1}(n)=2n \]
\paragraph{Zu zeigen:$\left|Pr\left[D(H_1(s))= 1\right] - Pr\left[D(r) = 1\right]\right| \leq negl(n) $}
Sei $s\in_R\{ 0,1 \}^n$ und $r\in_R\{ 0,1 \}^{2n}$
\korr{}{\[ \overline{~}:\{ 0,1 \}^n \rightarrow \{ 0,1 \}^n \] 
\[ b=\{ b_1,b_2,\ldots, b_n \} \mapsto \{ \overline{b_1},\overline{b_2},\ldots, \overline{b_n} \} \]
\[ \text{Da }s\in_R \{0,1\}^n \Rightarrow \overline{s}\in_R\{0,1\}^n \]	
}
\[ s\text{ zufällig gleichverteilt} \Rightarrow Pr\left[D(G_2(\overline s))=1\right] =Pr\left[D(G_2(s))= 1\right]\]
\[\Rightarrow Pr\left[D(G_2(\overline s))=1\right]=Pr\left[D(G_2(s))=1\right]=Pr\left[D(H_1(s))=1\right]\]
$\Rightarrow$ $H_1$ ist ein PZG.

\subsection*{b)}
\paragraph{Behauptung}
Sei $H_2(s) = G_1(s)|G_2(\overline s)$. Zeige, dass $H_2$ nicht notwendig ein PZG ist, indem du $G_1$ und $G_2$ so wählst, dass $H_2$ nicht wirklich randomisiert ist.
\paragraph{Gegenbeispiel}
Sei $G_1(s)$ ein PZG und $G_2(s) := G_1(\overline s)$ und somit auch ein PZG. Und $negl(n)$ die Menge aller vernachlässigbaren Funktionen.
\[ \Rightarrow H_1(s)=G_1(s)|G_2(\overline s) = G_1(s)|G_1(s) \]
$\Rightarrow$ Für einen Unterscheider $\mathcal D$, welcher in eine Bitfolge $s$ der Länge $4n$ überprüft, ob die ersten $\frac{4n}{2}$ den letzten $\frac{4n}{2}$ Bits entsprechen gilt:
\[ Pr\left[D(H_2(s))= 1\right] = 1 \land =Pr\left[D(r)= 1\right] = \frac{2^{2n}}{2^{4n}} = \frac{1}{2^{2n}} \]
\[\Rightarrow \left|Pr\left[D(H_2(s))= 1\right] - Pr\left[D(r) = 1\right]\right| = \frac{2^{2n}-1}{2^{2n}}\notin negl(4n)  \]
$\Rightarrow$ $H_2(s)$ ist kein PZG.
\begin{flushright}
	q.e.d.
\end{flushright}  


\end{document}


\documentclass[a4paper,11pt,twoside]{article}
\usepackage[T1]{fontenc}
\usepackage[utf8]{inputenc}
\usepackage{ngerman, eucal, mathrsfs, amsfonts, bbm, amsmath, amssymb, stmaryrd,graphicx, array, geometry, listings, color, ulem, epstopdf}
\geometry{left=25mm, right=15mm, bottom=25mm}
\setlength{\parindent}{0em} 
\setlength{\headheight}{0em} 
\title{Theoretische Grundlagen der Informatik II\\ Blatt 9}
\author{Markus Vieth \and Marvin Becker}
\date{\today}
\definecolor{comment_green}{rgb}{0, 0.5, 0}
\newcommand{\green}[1]{\textcolor{comment_green}{#1}}
\newcommand{\korr}[2]{\sout{#1} \textcolor{red}{\underline{#2}}}
\newcommand{\red}[1]{\textcolor{red}{#1}}

\lstset{%
	mathescape = true,%
	literate=%
	{Ö}{{\"O}}1
	{Ä}{{\"A}}1
	{Ü}{{\"U}}1
	{ß}{{\ss}}2
	{ü}{{\"u}}1
	{ä}{{\"a}}1
	{ö}{{\"o}}1
	{pi}{{$\Pi$}}1
}

\begin{document}

\maketitle
\cleardoublepage
\pagestyle{myheadings}
\markboth{Markus Vieth, Marvin Becker}{Markus Vieth, Marvin Becker}

\section*{Aufgabe 1}
Siehe Grafik 1
\begin{flushright}
	q.e.d.
\end{flushright}
\section*{Aufgabe 2}
\subsection*{a)}
\[ T\cup e' \text{ erzeugt Kreis }C \]
\[ T'\text{ ist Spannbaum} \Rightarrow \exists e\in (T\setminus T')\cap C \]
da sonst Kreis auch in T'
\subsection{b)}
Nach dem Lemma aus Aufgabe a) kann man eine Kante e aus $T$ mit einer Kante e' aus $T'$ tauschen, so dass ein neuer Spannbaum entsteht. Seien $T_i:= T_{i-1}.swap(e,e') | i\in\{1,\ldots,k\},~~T_0=T,~~e\in T_{i-1}\setminus T', e'\in T'\setminus T_{i-1}$ die Spannbäume, welche nach i Schritten entstehen. Weiter gilt $|T\setminus T'| - i = |T_i\setminus T'|$, da mit jedem Schritt sich der Spannbaum $T_i$ dem Spannbaum $T'$ annähert. Sobald $i = |T\setminus T'|$ folgt $T_i = T'$, da sich keine Kante mehr unterscheidet. Die maximale anzahl an Schritten ist somit $|T\setminus T'| \leq n-1$, da sich höchstens um $n-1$ Kanten, bei 2 Spannbäumen mit disjunkten Kantenmengen, unterscheiden können.
\end{document}


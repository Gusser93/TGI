\documentclass[a4paper,11pt,twoside]{scrartcl}
\usepackage{ngerman, eucal, mathrsfs, amsfonts, bbm, amsmath, amssymb, stmaryrd,graphicx, array, xcolor, ulem, epstopdf}
\usepackage[T1]{fontenc}
\usepackage[utf8]{inputenc}
\usepackage{geometry}
\geometry{left=25mm, right=15mm, bottom=25mm}
\setlength{\parindent}{0em} 
\setlength{\headheight}{0em} 
\newcommand{\korr}[2]{\sout{#1} \textcolor{red}{\underline{#2}}}
\usepackage{xcolor}
\usepackage{listings}

\definecolor{comment_green}{rgb}{0, 0.5, 0}
\newcommand{\green}[1]{\textcolor{comment_green}{#1}}
\newcommand{\red}[1]{\textcolor{red}{#1}}

\lstset{%
	mathescape = true,%
	literate=%
	{Ö}{{\"O}}1
	{Ä}{{\"A}}1
	{Ü}{{\"U}}1
	{ß}{{\ss}}2
	{ü}{{\"u}}1
	{ä}{{\"a}}1
	{ö}{{\"o}}1
	{pi}{{$\Pi$}}1
}

\title{Theoretische Grundlagen der Informatik II\\ Blatt 11}
\author{Markus Vieth \and Marvin Becker}
\date{\today}

\newcounter{beweis}

\begin{document}

\maketitle
\cleardoublepage
\pagestyle{myheadings}
\markboth{Markus Vieth, Marvin Becker}{Markus Vieth, Marvin Becker}

\section*{Aufgabe 1}
Nach dem Theorem aus der Vorlesung (Foliensatz 8, Folie 13), gilt:
\[ ((Gen,Enc,Dec)\text{ ist perfekt sicher} \Rightarrow |K|\geq |M|) \Leftrightarrow (|K|<|M|\Rightarrow (Gen,Enc,Dec)\text{ ist nicht perfekt sicher})  \]
Laut Vorlesung gilt für das normale One-Time-Pad $K=M\Rightarrow |K|=|M|$.\\
Nach der Aufgabenstellung gilt $|K|=|M|>|K|-1=|K'|$, wobei $K'$ die geänderte Schlüsselmenge aus der Aufgabenstellung meint (ohne $0^\ell$).\\
$\Rightarrow$ Das in der Aufgabenstellung beschriebene Verfahren kann nicht perfekt sicher sein.
\section*{Aufgabe 2}
\subsection*{a)}
$E_1$ kann nicht sicher sein, da wir für $2^n$ Nachrichten nur $1$ Schlüssel verwenden. Da Nachrichten mindestens die Länge 1 besitzen gilt $2^n>1$. Somit ist das Verfahren laut dem Theorem aus der Vorlesung nicht sicher.
\subsection*{b)}
Da die $0$ keinerlei Aussage über die Verschlüsselung oder die Nachricht selbst liefert, bleibt das Verfahren durch die Verwendung von $\Pi$ sicher.
\subsection*{c)}
Da wir davon ausgehen, dass der Angreifer $Enc$ aus $\Pi$ kennt, braucht dieser lediglich den Key von der Nachricht zu trennen, um letztere entschlüsseln zu können.
\subsection*{d)}
In dem man das Bit $m_n$ seiner Testnachtrichten einmal 0 und einmal 1 wählt, kann der Angreifer diese später mit einer Wahrscheinlichkeit von $100\%$ unterscheiden, da dieser nur das erste Bit des Geheimtextes mit dem letzten Bit seiner Nachricht vergleichen muss.
\subsection*{e)}
Dieses Verfahren muss sicher sein. Wäre es nicht sicher, so wäre $\Pi$ nicht sicher, da man das Verfahren zur Unterscheidung bei $E_5$ bei $\Pi$ anwenden könnte, in dem man die durch $\Pi$ verschlüsselte Nachricht einfach nochmal an dessen Ende kopiert und dann jenes Verfahren anwendet. Da aber $\Pi$ sicher ist, muss auch $E_5$ sicher sein.
\pagebreak
\section*{Aufgabe 3}
Beweis gilt unter der Annahme, das Eve weiß, dass in der Nachricht "`To: Bob"' steht.\\
Es gelte die ASCII Kodierung mit 8-Bit Erweiterung (sollte dies nicht gewünscht sein, so ist lediglich die 0 in Klammern am Anfang eines jeden Byteblocks zu entfernen).\\
Es sei $m$ die unverschlüsselte Originalnachricht und $n$ die von Eve gewünschte Nachricht.
Weiter sei $u$ die Zahl der Bits, welche vor der Passage "`To: Bob"' in der Nachricht stehen und $v$ die Zahl der Bits nach der Passage. Des Weiteren soll gelten:
\[ m^*:=0^u||(0)1010100~(0)1101111~(0)0111010~(0)0100000~(0)1000010~(0)1101111~(0)1100010||0^v \] 
\[ n^*:=0^u||(0)1010100~(0)1101111~(0)0111010~(0)0100000~(0)1000101~(0)1110110~(0)1100101||0^v \]
Somit entspricht $m^*$ der Passage "`To: Bob"' an der entsprechenden Stelle, ohne den Rest der Nachricht und $n^*$ das selbe für "`To: Eve"'\\
Wählt Eve nun als "`p"':
\[p:=m^*\oplus n^*= 0^u||(0)0000000~(0)0000000~(0)0000000~(0)0000000~(0)0000111~(0)0011001~(0)0000111||0^v \]
So gilt:
\[ m\oplus p = n \]
Das heißt für den Angriff:
\[ ((m\oplus k)\oplus p)\oplus k =\footnote{Aus der technischen Informatik ist bekannt, dass XOR assoziativ und kommutativ ist.} m\oplus k\oplus p \oplus k = k\oplus k \oplus m \oplus p = 0 \oplus n = n \]
$\Rightarrow$ Mit $p$ lässt sich die Nachricht auf die gewünschte Art manipulieren, trotz perfekt sicherer Verschlüsselung.
\begin{flushright}
	q.e.d.
\end{flushright}
\end{document}


\documentclass[a4paper,11pt,twoside]{article}
\usepackage[T1]{fontenc}
\usepackage[utf8]{inputenc}
\usepackage{ngerman, eucal, mathrsfs, amsfonts, bbm, amsmath, amssymb, stmaryrd,graphicx, array, geometry, listings, color}
\geometry{left=25mm, right=15mm, bottom=25mm}
\setlength{\parindent}{0em} 
\setlength{\headheight}{0em} 
\title{Theoretische Grundlagen der Informatik II\\ Blatt 3}
\author{Markus Vieth, David Klopp, Christian Stricker}
\date{\today}

\begin{document}

\maketitle
\cleardoublepage
\pagestyle{myheadings}
\markboth{Markus Vieth,  David Klopp, Christian Stricker}{Markus Vieth, David Klopp, Christian Stricker}

\section*{Aufgabe 1}

\subsection*{a)}
\underline{Behauptung:} A ist NP-vollständig und $ B \in NP $. Wenn  $ A \leq_{poly} $ B, dann ist B auch NP-vollständig.
\newline
\underline{Beweis:} A ist NP-vollständig => A ist NP-hart => Da man A auf B in polynomieller Zeit reduzieren kann, ist B auch NP-hart. Da B in NP liegt, folgt daraus: B ist NP-vollständig.

\subsection*{b)}
\underline{Behauptung:} Sei A ein NP-hartes Problem. Wenn  $ A \leq_{poly} B $ , dann ist B in der Komplexitätsklasse P.
\newline
\underline{Widerspruchsbeweis:} Angenommen wahr: A ist NP-hart => $ X \leq_{poly} A ~ \forall ~ X \in NP => X \leq_{poly} A \leq_{poly} B ~ \forall ~ X \in NP => X \leq_{poly} B ~\forall~ X \in NP, da B \in P => NP = P ~ Widerspruch. $

\subsection*{c)}
\underline{Behauptung:} Sei A ein Problem. Sei B ein NP-hartes Problem. Wenn A in der Komplexitätsklasse P ist, dann gibt es eine Reduktion $ A \leq_{poly} B$.
\newline
\underline{Beweis:} $ A \in P  \subset NP =>  A \leq_{poly} B $

\subsection*{d)}
\underline{Behauptung:} Für ein beliebiges Problem X gelte $ SAT \leq_{poly} X \leq_{poly} SAT$ . Dann ist X NP-vollständig.
\newline
\underline{Beweis:} SAT ist ein NP-vollständiges Problem. Da SAT auf X in polynomieller Zeit reduziert werden kann, ist SAT NP-hart. 
\newline
$SAT \in NP $ und $ X \leq_{poly} SAT $ => $ X \in NP $ . DA $ X \in NP $ und X NP-hart => X ist NP-vollständig 

\subsection*{e)}
\underline{Definition Klasse NP:} Menge aller Entscheidungsprobleme, die effizient überprüft werden können.
\newline
Es existiert ein polynomieller Algorithmus B mit zwei Eingaben s und t und ein Polynom p, so dass gilt:
\newline Ist $ s \notin X $ , so ist B(s,t) = no für alle $ t \in \{0,1\}^* $
\newline Ist $ s \in X $, so existiert rin $ t \in \{0,1\}^* $ mit $ |t| \leq p(s|)$ und B(s,t) = yes
\newline

\subsection*{f)}
\underline{Satz von Cook und Levin:} SAT ist NP-vollständig und wenn ein Problem X auf SAT reduziert werden kann, kann man auch das Problem X auf SAT reduzieren.
\newline

\subsection*{g)}
Unter der Annahme, das die Reduktion den Algorithmus von A nicht beinhaltet:\\
Da der Algorithmus B durch A gelöst wird, beträgt die Laufzeit die Summe von A und der Reduktion: $ O(n^5+n^4) = O(n^5) $ 

\section*{Aufgabe 2}
\subsection*{Behauptung: Set-Cover(SC) ist NP-Vollständig}
\subsection*{Zu zeigen: Set-Cover(SC) $\in$ NP}
\subsubsection*{Guess}
Es werden nichtdeterministisch k paarweise verschiedene Teilemengen S ausgewählt und in T gespeichert.
\subsubsection*{Check}
kopiere M in M'. (Laufzeit |M|=n)\\
für alle Teilmengen $S_i$ in T (Laufzeit höchstens n)\\
$~~~$für alle Elemente $e_j\in S_i$ (Laufzeit höchstens n)\\
$~~~~~~$für alle Elemente $e_k\in M'$ (Laufzeit höchstens n)\\
$~~~~~~~~~$wenn $e_j = e_k$ (Laufzeit konstant)\\
$~~~~~~~~~~~~$lösche $e_k$ in M' (Laufzeit konstant)\\
Wenn $|M'|=0$ \\
$~~~$ return ja \\
sonst\\
$~~~$ return nein.
\[\Rightarrow T(n)\in O(n^3)\]
$\Rightarrow SC\in NP$
\subsection*{Zu zeigen: Set-Cover(SC) ist NP-hart}
Für jeden Knoten im Graphen existiert genau eine Teilemenge. (Laufzeit n)
\[\forall v_i \in V \exists! S_i\]
Für jede Kante im Graphen existiert genau ein Element in M. (Laufzeit bis zu $n^2$)
\[\forall k_i \in E \exists! e_i \in M\]
Die Elemente, welche eine Kante repräsentieren, befinden sich genau dann in einer Teilemenge eines Knoten, wenn die Kante am Knoten endet.
\[e_i\in S_j \Leftrightarrow \exists k_i=\{v_i,v_x\} \text{mit beliebigen x}\]
für alle Kanten $k_i\in E$ (Laufzeit bis zu $n^2$)\\
$~~~$für alle Knoten $v_j\in k_i$ (Laufzeit 2)\\
$~~~~~~$speichere $e_i$ in $S_j$
\[\Rightarrow T(n)\in O(n^2)\]
\[\Rightarrow \text{SC ist NP-Hart} \land SC\in NP \Rightarrow \text{SC ist NP-Vollständig}\]

\end{document}
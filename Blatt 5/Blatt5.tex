\documentclass[a4paper,11pt,twoside]{article}
\usepackage[T1]{fontenc}
\usepackage[utf8]{inputenc}
\usepackage{ngerman, eucal, mathrsfs, amsfonts, bbm, amsmath, amssymb, stmaryrd,graphicx, array, geometry, listings, color}
\geometry{left=25mm, right=15mm, bottom=25mm}
\setlength{\parindent}{0em} 
\setlength{\headheight}{0em} 
\title{Theoretische Grundlagen der Informatik II\\ Blatt 5}
\author{Markus Vieth, David Klopp, Christian Stricker}
\date{\today}

\begin{document}

\maketitle
\cleardoublepage
\pagestyle{myheadings}
\markboth{Markus Vieth,  David Klopp, Christian Stricker}{Markus Vieth, David Klopp, Christian Stricker}

\section*{Aufgabe 1}


\section*{Aufgabe 2}
\underline{Annahme:}  $P \neq NP$, da SAT nur mit einem exponentiellen Aufwand gelöst werden kann und SAT in NP liegt.\\
\underline {Widerspruch:} Um zu zeigen, dass NP = P gilt, reicht es aus, ein Algorithmus zu finden, der ein NP-hartes Problem in P löst.\\
Der Gegenbeweis wäre, es gibt keinen einzigen Algorithmus, der ein NP-hartes Problem in P löst. \\
Daraus folgt, es reicht längst nicht aus, nur einen einzigen Algorithmus zu finden, für das gilt: NP = P.\\
Da wahrscheinlich bisher noch nicht alle Algorithmen gefunden wurden, kann man nicht für jeden Algorithmus zeigen, dass $NP \neq P$ gilt und somit ist die Annahme hinfällig.



\section*{Aufgabe 3}


\end{document}
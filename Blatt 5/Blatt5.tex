\documentclass[a4paper,11pt,twoside]{article}
\usepackage[T1]{fontenc}
\usepackage[utf8]{inputenc}
\usepackage{ngerman, eucal, mathrsfs, amsfonts, bbm, amsmath, amssymb, stmaryrd,graphicx, array, geometry, listings, color, ulem}
\geometry{left=25mm, right=15mm, bottom=25mm}
\setlength{\parindent}{0em} 
\setlength{\headheight}{0em} 
\title{Theoretische Grundlagen der Informatik II\\ Blatt 5}
\author{Markus Vieth, David Klopp, Christian Stricker}
\date{\today}
\newcommand{\korr}[2]{\sout{#1} \textcolor{red}{\underline{#2}}}
\newcommand{\red}[1]{\textcolor{red}{#1}}
\definecolor{command_green}{rgb}{0, 0.5, 0}
\newcommand{\green}[1]{\textcolor{command_green}{#1}}

\begin{document}

\maketitle
\cleardoublepage
\pagestyle{myheadings}
\markboth{Markus Vieth,  David Klopp, Christian Stricker}{Markus Vieth, David Klopp, Christian Stricker}

\section*{Aufgabe 1}
\subsection*{Beh: INDEPENDENT-SET $\in$ NP:}
\subsection*{Bew:}
\subsubsection*{GUESS}
Rate nicht-deterministisch eine Menge von Knoten der Größe k.

\subsubsection*{CHECK}
vorhandene\_kanten = []\\
FOR v IN V: \green{// iteriert über alle Knoten} \\
$~~~~~~~~$FOR k IN v.getKanten(): \green{// iteriert über alle Kanten eines Knotens} \\
$~~~~~~~~~~~~~~~~$FOR i IN vorhandene\_kanten: \green{// iteriert über alle Kanten des Arrays}\\
$~~~~~~~~~~~~~~~~~~~~~~~~$IF i == k: \\
$~~~~~~~~~~~~~~~~~~~~~~~~~~~~~~~~$return Nein \green{// zwei Knoten die zu einer Kante passen} \\
$~~~~~~~~~~~~~~~~~~~~~~~~$ELSE:\\
$~~~~~~~~~~~~~~~~~~~~~~~~~~~~~~~~$vorhandene\_kanten.add(k) \\
return Ja \green{// jede Kante der Lösung ist nur einmal vertreten}


\subsection*{Laufzeit:}
\subsubsection*{GUESS:}
Im worst-case, also für k=n beträgt die Laufzeit n.

\subsubsection*{CHECK:}
Falls alle Knoten untereinander verbunden sind, beträgt die Laufzeit zum Durchlaufen aller Kanten: $n \cdot \sum_{i=0}^{n-1} i$. Nach Gauß folgt daraus: $n \cdot \frac{(n^2-n)}{2}$. 

\subsubsection*{Gesamt:}
Die gesamt Laufzeit beträgt somit: $n \cdot \frac{(n^2-n)}{2} + n \in O(n^3)$ \\
$q.e.d.$  


\subsection*{Reduktion: CLIQUE $\leq_p $ INDEPENDENT-SET}
\subsubsection*{Vorgehen:}
Verbinde alle Knoten untereinander, die durch keine Kante verbunden sind. Lösche alle ursprünglichen Kanten und wende nun das INDEPENDENT-SET Problem mit der Größe k an. \\

\subsubsection*{Formal:} 
Eingabe: Graph G(V,E), ganze Zahl k \\

$E' := \{<v,w> | \forall~ v \not= w ~and ~v,w \in V \} \setminus E$\\
$G' := G(V, E')$\\
$ INDEPENDENT-SET(G', k)$
 
 \subsubsection*{Laufzeit:} 

1. Bilde die Menge alle Kante $\Rightarrow \sum_{i=1}^n(n-i) = \sum_{i=0}^{n-1} i = \frac{n^2-n}{2}$ \\
2. Entferne aus der Menge aller möglichen Kanten jene, die in E liegen. Iteriere hierzu über die Menge aller möglichen Kanten, so wie über E $\Rightarrow (\frac{n^2-n}{2})^2$ \\
$\Rightarrow$ Laufzeit: $O(n^4)$


 \subsubsection*{Äquivalenzbeweis:} 
'' $\Rightarrow$ ''
Geg: CLIQUE\\
Alle Knoten in einer Clique sind untereinander verbunden. Durch die Komplementbildung der Kanten sind diese Knoten nun alle nicht mehr miteinander verbunden. Es ergibt sich somit ein Independent-Set. \\

'' $\Leftarrow$ ''
Geg: Independent-Set\\
Alle Knoten eines Independent-Set weisen untereinander keine Kanten auf. Nach der Bildung des Komplements sind alle Knoten untereinander verbunden. Es ergibt sich eine Clique. \\

$q.e.d$


\section*{Aufgabe 2}
\underline{Annahme:}  $P \neq NP$, da SAT nur mit einem exponentiellen Aufwand gelöst werden kann und SAT in NP liegt.\\
\underline {Widerspruch:} Um zu zeigen, dass NP = P gilt, reicht es aus, ein Algorithmus zu finden, der ein NP-hartes Problem in P löst.\\
Der Gegenbeweis wäre, es gibt keinen einzigen Algorithmus, der ein NP-hartes Problem in P löst. \\
Daraus folgt, dass es längst nicht ausreicht, nur einen einzigen Algorithmus zu finden, für den gilt: NP = P.\\
Da wahrscheinlich bisher noch nicht alle Algorithmen gefunden wurden, kann man nicht für jeden Algorithmus zeigen, dass $NP \neq P$ gilt,womit  die Annahme hinfällig ist.


\pagebreak
\section*{Aufgabe 3}
\subsection*{Vorüberlegung:}
\[P=NP \Rightarrow SAT\in NP=P\Rightarrow SAT\in P\]
\subsection*{zu Zeigen:}
Es existiert ein Algorithmus, welcher für eine erfüllbare aussagenlogische Formel $\Phi$ in polynomieller Zeit eine erfüllende Belegung ausgibt.\\
\subsection*{Beweis:}
Sei die Laufzeit des polynomiellen SAT-Algorithmus = p\\
Sei $X:=\{x_i| \forall x_i,\overline{x_i} \in \Phi\}$ die Liste aller Literale in $\Phi$, |X| ist im worst-case = n.\\
if (SAT($\Phi(X)$)\{  //Teste ob $\Phi$ erfüllbar ist; (Laufzeit ist p) \\
$~~~$for(int i=0; i < |X|; i++) \{  //Setze jedes $x_i$ auf einen festen Wert, (im worst-case n Durchläufe)\\
$~~~$$~~~$X[i]=0;  //Teste ob $\Phi$ erfüllbar, wenn $x_i=0$\\
$~~~$$~~~$if(!(SAT($\Phi(X)$))) \{ //Wenn nicht: (Laufzeit = p)\\
$~~~$$~~~$$~~~$X[i]=1; 	//, dann muss $x_i$ gleich 1 sein\\
$~~~$$~~~$\} //Wenn SAT mit $x_i=0$ erfüllbar ist, bleibt $x_i$ gleich 0 \\
$~~~$\} //Wenn dies für alle $x_i$ erfolgreich durchgeführt wurde, besitzt X eine erfüllende Belegung\\
$~~~$ String result ="'"';\\
$~~~$ for (int i = 0; i < |X| ; i++)\\
$~~~$$~~~$result = result+X[i]+"'"';\\
$~~~$return result;\\
\}\\
else\\
$~~~$return "'nicht erfüllbar"';\\
\\
Die erste Überprüfung dauert p lange, die folgende Schleife hat im worst-case n Durchläufe und in jedem Durchlauf einen SAT-Aufruf mit der Laufzeit p, somit beträgt die Laufzeit $T(n)=p+p\cdot n \Rightarrow$ die Laufzeit des oben genannten Algorithmus ist selbst wieder polynomiell, da die Laufzeit höchstens ein Polynom vom Grad $grad(p)+1$ ist. Das die Ausgabe ein legitimes Ergebnis ist, ist trivial und geht aus dem Algorithmus hervor.
\end{document}
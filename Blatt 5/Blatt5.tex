\documentclass[a4paper,11pt,twoside]{article}
\usepackage[T1]{fontenc}
\usepackage[utf8]{inputenc}
\usepackage{ngerman, eucal, mathrsfs, amsfonts, bbm, amsmath, amssymb, stmaryrd,graphicx, array, geometry, listings, color, ulem}
\geometry{left=25mm, right=15mm, bottom=25mm}
\setlength{\parindent}{0em} 
\setlength{\headheight}{0em} 
\title{Theoretische Grundlagen der Informatik II\\ Blatt 5}
\author{Markus Vieth, David Klopp, Christian Stricker}
\date{\today}
\newcommand{\korr}[2]{\sout{#1} \textcolor{red}{\underline{#2}}}
\newcommand{\red}[1]{\textcolor{red}{#1}}

\begin{document}

\maketitle
\cleardoublepage
\pagestyle{myheadings}
\markboth{Markus Vieth,  David Klopp, Christian Stricker}{Markus Vieth, David Klopp, Christian Stricker}

\section*{Aufgabe 1}


\section*{Aufgabe 2}



\section*{Aufgabe 3}
\subsection*{Vorüberlegung:}
\[P=NP \Rightarrow SAT\in NP=P\Rightarrow SAT\in P\]
\subsection*{zu Zeigen:}
Es existiert ein Algorithmus, welcher für eine erfüllbare aussagenlogische Formel $\Phi$ in polynomieller Zeit eine erfüllende Belegung ausgibt.\\
\subsection*{Beweis:}
Sei die Laufzeit des polynomiellen SAT-Algorithmus = p
Sei $X:=\{x_i| \forall x_i \in \Phi\}$ die Liste aller Literale in $\Phi$, |X| ist im worst-case = n.\\
if (SAT($\Phi(X)$)\{  //Teste ob $\Phi$ erfüllbar ist; (Laufzeit ist p) 
$~~~$for(int i=0; i < |X|; i++) \{  //Setze jedes $x_i$ auf einen festen Wert, (im worst-case n Durchläufe)
$~~~$$~~~$X[i]=0;  //Teste ob $\Phi$ erfüllbar, wenn $x_i=0$
$~~~$$~~~$if(!(SAT($\Phi(X)$))) \{ //Wenn nicht: (Laufzeit = p)
$~~~$$~~~$$~~~$X[i]=1; 	//, dann muss $x_i$ gleich 1 sein
$~~~$$~~~$\} //Wenn SAT mit $x_i=0$ erfüllbar ist, bleibt $x_i$ gleich 0 
$~~~$\} //Wenn dies für alle $x_i$ erfolgreich durchgeführt wurde, besitzt X eine erfüllende Belegung
$~~~$ String result ="'"';
$~~~$ for (int i = 0; i < |X| ; i++)
$~~~$$~~~$result = result+X[i]+"'"';
$~~~$return result;
\}
else
$~~~$return "'nicht erfüllbar"';

Die erste Überprüfung dauert p lange, die folgende Schleife hat im worst-case n Durchläufe und in jedem Durchlauf einen SAT-Aufruf mit der Laufzeit p, somit beträgt die Laufzeit $T(n)=p+p\cdot n \Rightarrow$ die Laufzeit des oben genannten Algorithmus ist selbst wieder polynomiell, da die Laufzeit höchstens ein Polynom vom Grad $grad(p)+1$ ist. Das die Ausgabe ein legitimes Ergebnis ist, ist trivial und geht aus dem Algorithmus hervor.
\end{document}
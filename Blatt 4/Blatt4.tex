\documentclass[a4paper,11pt,twoside]{article}
\usepackage[T1]{fontenc}
\usepackage[utf8]{inputenc}
\usepackage{ngerman, eucal, mathrsfs, amsfonts, bbm, amsmath, amssymb, stmaryrd,graphicx, array, geometry, listings, color}
\geometry{left=25mm, right=15mm, bottom=25mm}
\setlength{\parindent}{0em} 
\setlength{\headheight}{0em} 
\title{Theoretische Grundlagen der Informatik II\\ Blatt 3}
\author{Markus Vieth, David Klopp, Christian Stricker}
\date{\today}

\begin{document}

\maketitle
\cleardoublepage
\pagestyle{myheadings}
\markboth{Markus Vieth,  David Klopp, Christian Stricker}{Markus Vieth, David Klopp, Christian Stricker}

\section*{Aufgabe 1}



\section*{Aufgabe 2}
Sei $f(x_1, ..., x_n)$ ein boolscher Ausdruck mit den Klauseln $Y_1, ..., Y_k$. 

\subsection*{Behauptung: SAT $\in$ NP}
\subsubsection*{Guess}
Es werden nichtdeterministisch die Wahrheitsbelegungen von $x_1$ bis $x_n$ geraten.
\subsubsection*{Check}

For $Y_i$ in $f(x_1,.., x_n)$:\\
$~~~$KlauselIstWahr = true\\
$~~~$For $x_i$ in $Y_i$:\\
\textit{// Falls ein Term 0 ist, wird die ganze Klausel false}\\
$~~~~~~$If $x_i == 0$:\\ 
$~~~~~~~~~$KlauselIstWahr = false\\
\textit{//Wenn eine Klausel wahr ist, ist der ganze boolsche Ausdruck wahr}\\
$~~~$If KlauselIstWahr == true:\\
$~~~~~~$return Ja\\
return Nein\\

Im worst-case beträgt die Laufzeit O(k*n) => $SAT \in NP$



\section*{Aufgabe 3}

\end{document}
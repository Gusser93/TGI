\documentclass[a4paper,11pt,twoside]{article}
\usepackage[T1]{fontenc}
\usepackage[utf8]{inputenc}
\usepackage{ngerman, eucal, mathrsfs, amsfonts, bbm, amsmath, amssymb, stmaryrd,graphicx, array, geometry, listings, color}
\geometry{left=25mm, right=15mm, bottom=25mm}
\setlength{\parindent}{0em} 
\setlength{\headheight}{0em} 
\title{Theoretische Grundlagen der Informatik II\\ Blatt 4}
\author{Markus Vieth, David Klopp, Christian Stricker}
\date{\today}

\begin{document}

\maketitle
\cleardoublepage
\pagestyle{myheadings}
\markboth{Markus Vieth,  David Klopp, Christian Stricker}{Markus Vieth, David Klopp, Christian Stricker}

\section*{Aufgabe 1}
\subsection*{1. INTEGER PROGRAMMING}
Eingabe: Matrix C aus ganzen Zahlen und ein ganzzahliger Vektor d \\
Beschreibung: Es gibt einen Vektor x bestehend aus 0 und 1, so dass Cx = d gilt.\\

\subsection*{2. CLIQUE:}
Eingabe: Graph G, positive ganze Zahl k\\
Beschreibung: Es gibt eine Menge der Größe k von Knoten, die alle paarweise untereinander verbunden sind.\\

\subsection*{3. SET PACKING:}
Eingabe: Familie von Mengen \{$S_j$\}, positive ganze Zahl l\\
Beschreibung: Beinhaltet \{$S_j$\} l paarweise disjunkte Teilmengen.\\

\subsection*{4. NODE COVER:}
Eingabe: Graph $G'$, positive ganze Zahl l\\\
Beschreibung: Es gibt eine Knotenmenge der Größe l, sodass alle Kanten mit mindestens einem Knoten aus dieser Menge verbunden sind.\\

\subsection*{5. SET COVERING:}
Eingabe: endliche Familie von endlichen Mengen \{$S_j$\}, positive ganze Zahl k\\
Beschreibung: Es gibt eine Unterfamilie  $\{T_h\} \subseteq \{S_j\}$ mit $\leq$ k Mengen, sodass $\cup T_h = \cup S_j$.\\

\subsection*{6. UNDIRECTED HAMILTON CIRCUIT:}
Eingabe: Graph G\\
Beschreibung: G enthält einen Zirkel, der jeden Knoten genau einmal beinhaltet.\\

\subsection*{7. DIRECTED HAMILTON CIRCUIT:}
Eingabe: gerichteter Graph H\\
Beschreibung: G enthält einen gerichteten Zirkel, der jeden Knoten genau einmal beinhaltet. \\

\subsection*{8. CLIQUE COVER:}
Eingabe: Graph $G'$, positive ganze Zahl l\\
Beschreibung: $N'$ ist die Vereinigung von l oder weniger Cliquen.\\

\subsection*{9. Knapsack:}
Eingabe: $(a_1, a_2, ..., a_r, b) \in Z^{n+1}$\\
Beschreibung: $\sum a_j x_j = b$ hat eine 0-1 Lösung.\\

\subsection*{10. PARTITION:}
Eingabe: $(c_1, c_2, ..., c_s) \in Z^{s}$\\
Beschreibung: Es gibt eine Menge $I \subseteq \{1,2,...,s\}$, sodass $\sum_{h \in I} c_h = \sum_{h \notin I} c_h$.\\


\section*{Aufgabe 2}
Sei $f(x_1, ..., x_n)$ ein boolscher Ausdruck mit den Klauseln $Y_1, ..., Y_k$. 

\subsection*{Behauptung: SAT $\in$ NP}
\subsubsection*{Guess}
Es werden nichtdeterministisch die Wahrheitsbelegungen von $x_1$ bis $x_n$ geraten.
\subsubsection*{Check}

For $Y_i$ in $f(x_1,.., x_n)$:\\
$~~~$KlauselIstWahr = true\\
$~~~$For $x_i$ in $Y_i$:\\
\textit{// Falls ein Term 0 ist, wird die ganze Klausel false}\\
$~~~~~~$If $x_i == 0$:\\ 
$~~~~~~~~~$KlauselIstWahr = false\\
\textit{//Wenn eine Klausel wahr ist, ist der ganze boolsche Ausdruck wahr}\\
$~~~$If KlauselIstWahr == true:\\
$~~~~~~$return Ja\\
return Nein\\

Im worst-case beträgt die Laufzeit O(k*n) => $SAT \in NP$



\section*{Aufgabe 3}
\subsection*{Reduktion 3-SAT $\le$ CLIQUE}
Seien $Y_1, ... Y_n$ die Klauseln des boolschen Ausdrucks.

\subsubsection*{Graph bilden:}
Bilde die Menge aller Knoten V:\\
V = \{$<\sigma, i>$ | wobei $\sigma$ ein Literal aus der Klausel $Y_i$ meint \}\\
D.h jedes Literal pro Klausel ist eindeutig durch einen Knoten vertreten.\\

Bilde die Menge der Kanten E:\\
E = \{\{$<\sigma, i>, <\delta, j>$\} $i \not= j \lor \sigma \not= \delta$ \}
D.h verbinde alle Kanten miteinander, die nicht in der selben Klausel sind und bei denen das Literal nicht der Negation des Literals entspricht.\\

Es ergibt sich somit der Graph G(V,E). Wähle k = $\#\{Y_1, ... Y_n\}$. D.h k entspricht der Anzahl von Klauseln des boolschen Ausdrucks.\\
Diese Operationen lassen sich in poylnomieller Laufzeit ausführen.\\

Wende nun den CLIQUEN-Algorithmus an. Die Lösung von 3-SAT ist nun äquivalent zur Lösung von CLIQUE.\\

$q.e.d$

\end{document}